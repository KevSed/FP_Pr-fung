\section{V01 Lebensdauer kosmischer Myonen}
\label{sec:myonen}

\begin{itemize}
    \item Bestimmung der Lebensdauer aus Einzellebensdauermessungen
    \item Myonen stammen aus Pionzerfällen in der Atmosphäre
    \item Zeitdilatation durch hohe Geschwindigkeit
    \item Nachweis über Szintillator (MeV Energiedeposition): Anregung des Szintillatormaterials
    \item Photonemission bei Abregung (sichtbares Licht)
    \item Bei Zerfall des Myons nochmalige Anregung durch hochenergetisches Elektron
    \item Problem: Unterscheide zufälliges Start und zugehöriges Stoppsignal
    \item Unabhängige Prozesse: Messung über Individuallebensdauern möglich
    \item $\text{d}W = \lambda\text{d}t$, also $\text{d}N = -N\text{d}W = -\lambda N\text{d}t$
    \item Große Messzahl macht Zusammenhang kontinuirlich: $\frac{\text{d}N}{N_0} = \lambda\text{e}^{-\lambda t}\text{d}t$
    \item Messe N und bestimme daraus $\lambda = \frac{1}{\tau}$ die Lebensdauer
\end{itemize}

\textbf{Durchführung}:
\begin{itemize}
    \item Organischer Szintillator (kurze Lebensdauern der angeregten Zustände sorgen für gute Zeitauflösung)
    \item SEV an beiden Enden: Photonen erzeugen Elektronen in PM, welche dann zu Signal verstärkt werden
    \item Verzögerungsleitungen an beiden Seiten: Ausgleich der bauteilbedingten Laufzeitunterschiede auf beiden Seiten
    \item Diskriminatoren: Signalschwelle (Rauschunterdrückung), sowie Umwandlung in logisches binäres Signal
    \item Koinzidenzschaltung: Impulse in Zeitintervall $\symup{\Delta}t$ werden als gleichzeitig gewertet und weitergeleitet. Sehr gute Rauschunterdrückung, da zwei Signale so durch Rauschen sehr unwahrscheinlich.
    \item deutlich mehr Start als Stop Impulse erwartet: begrenze Wartezeit auf Stopimpuls
    \item Koinzidenzausgang aktiviert START am \textit{time amplitude converter} (digitalisierung des Signals in Zeitkanäle)
    \item Kippstufe am Suchzeitgeber aktiviert: Signal an AND2
    \item Wenn in diesem Zeitraum zweiter Zerfall, wird Zeit gemessen
    \item sonst Kippstufe wieder in Anfangszustand
    \item Diskriminatoren mit Oszilloskop so einstellen, dass einheitliche Breiten (wieso?) und Höhen der Impulse
    \item Kalibrierung der Verzögerung auf maximalen Fluss
    \item Kalibrierung \textit{tac} durch Doppelimpulsgenerator
\end{itemize}

\textbf{Auswertung:}
\begin{itemize}
    \item Koinzidenz: $\SI{20}{\nano\second}$ breiten $\SI{1.3}{\volt}$ hohe Impulse
    \item optimale Verzögerung aus Maximum des Flusses bei $\SI{5,5}{\nano\second}$
    \item Zeitkalibrierung der Kanäle: Regression der Impulszeiten gegen den Kanal
    \item Abschätzung der Untergrundkandidaten über Poissonverteilung:
        \begin{equation}
            N_F = N\frac{(T_S\bar{N})^k}{k!}\text{e}^{T_S\bar{N}}
        \end{equation}
    \item daraus bei Gleichverteilung Untergrund pro Kanal
    \item Fit einer Exponentialfunktion mit gleichverteiltem Untergrund an die Messdaten
    \item Lebensdauer etwa $\SI{2,2}{\micro\second}$, Untergrund kleiner als erwartet
    \item Verbesserung vielleicht durch längere Messdauer bei Kalibrierung der Kanäle
\end{itemize}
