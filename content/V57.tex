\section{V57 Rauschen}
\label{sec:V57}

\begin{itemize}
    \item Untersuchung des thermischen Spannungsrauschens, sowie des Stromrauschens von zwei Kathoden
    \item Außerdem Rauschzahl und Boltzmann-Konstante
    \item Wechselspannungen aus Potentialdifferenzen durch Fluktuationen
    \item durchaus Auswirkungen auf Präzisionsmessungen
    \item unterschiedliches Frequenzverhalten
    \item thermisches Rauschen an Ohmschen Widerständen, Elektroden unterschiedliche Anzahl an austretenden Elektronen
    \item Abhängig von der Geometrie des Bauteils. \textbf{Schrot}-Effekt
    \item Effekte durch zeitliche Änderung der Austrittsarbeit des Anodenmaterials, \textbf{Funkel}-Effekt
    \item Da statistisch, verschwindet Mittelwert über große Zeiten, sodass Mittelwert der quadrierten Spannung untersucht werden muss
\end{itemize}

\textbf{Thermisches Rauschen}
\begin{itemize}
    \item Nyquist Beziehung: $\bar{U^2} = 4k_BTR\upD\nu$ beschreibt weißes Rauschen, also unabh. von Breite des Frequenzbandes
    \item Rauschspannungsquadrat der Widerstände durch endliche Kapzitäten verringert
    \item Maxwell-Boltzmann verteilt
\end{itemize}

\textbf{Schrotrauschen}
\begin{itemize}
    \item Elektronenröhren: Effekte bei Übertragung der Elektronen von Kathode zur Anode
    \item Schottky-Beziehung $\bar{I^2} = 2e_0I_0\upD\nu$, allerdings Annahmen, dass $v_0$ der Elektronen Null, keine ablenkenden Sekundärelektronen
\end{itemize}

\textbf{Funkel-Effekt}
\begin{itemize}
    \item $\sfrac{1}{f}$-Abhängigkeit, also wichtig bei kleinen Frequenzen
    \item gültig für sehr große Frequenzbereiche und ohne untere Grenze
    \item Funkel-Effekt bei Elektronenröhren mit Oxydkathode
    \item physikalische Effekte: atomare Diffusionsprozesse (lokale sprunghafte Änderung des Widerstandes)
    \item große Frequenzen dominiert von Schwankungen der Austrittsarbeit (durch lokale Fremdatome) der Oxydkathode
\end{itemize}

\textbf{Durchführung}
\begin{itemize}
    \item Spannungsrauschen zweier Ohmscher Widerstände
\end{itemize}
